\documentclass[a4paper,12pt]{report}
%\documentclass[a4paper,10pt]{scrartcl}

\usepackage[utf8]{inputenc}

\title{Reconstruction of a macro-complex using interacting subunits}
\author{Lydia Fortea \and Juan Luis Melero}
\date{}

\pdfinfo{%
  /Title    (Reconstruction of a macro-complex using interacting subunits)
  /Author   (Lydia Fortea and Juan Luis Melero)
  /Creator  ()
  /Producer ()
  /Subject  (Structural Bioinformatics and Introduction to Python)
  /Keywords (modelling, reconstruction, macro-complex, structural alignment, structural bioinformatics)
}

\begin{document}
\maketitle
\tableofcontents{}

\chapter{Background}

The aim of the project is to reconstruct a marco-complex having only the pair interacting chains, using a standalone program created by ourselves.
The program we created is based on several bioinformatic features, including modelling, structural superimposition and sequence alignment, among others.

\section{Protein-Protein Interaction and Complexes}

An important point of the project is understanding the Protein-Protein Interaction and Complexes. In the quaternary structure of a protein, there are more than one separated chains of proteins that interacti between them.
The interaction of these chains can involve a lot of intermolecular bounds, such as hydrogen bounds, electrostatic interactions, pi stacking, cation-pi interaction, etc. This diversity of interactions makes the protein-protein interaction
very common in order to stabilized the molecule and generate a biological function.\\\\

The whole structure, where two or more chains are combined and have one or different functions, is called a complex. The formation of a complex can be made by protein-protein interaction only or nucleotides (DNA or RNA) can also be part of a complex if there is DNA-protein or RNA-protein interactions.\\\\

Focusing on the project, having the protein-protein interaction by pairs, we want to reconstruct the whole macro-complex.

\section{Structural superimposition}

We cannot assume that the protein-protein interacting pairs are well oriented in the space. Therefore, in order to give to each part the correct orientation, we do a structural superimposition. 


%\section{Complex Modelling}

%Modelling is the process through which having a protein sequence and one or more templates we can infer the structure of the protein. This is made by a program called \textit{Modeller}. 



\chapter{Algorithm and Program}


\section{Inputs and Outputs}

\subsection{Input files}

There are two types of input files to the program:

\begin{itemize}
 \item PDB files with two interacting chains
 \item For each PDB files of the interaction of two chains, PDB files of each chain separatedly.
\end{itemize}

Input files must be all in the directory specified in the arguments or in the current directory by default. The program will read all the PDB files, si it is necessary that all the PDB files in the working directory are the subunits of the macrocomplex and nothing else.

\subsection{Output file}

The output file will be one PDB file in which there will be the coordinates of the atoms of the macro-complex. The output file will be located in the directory specified in the arguments. By default, it creates a file called \textit{output.pdb} in the current directory. If the current directory is the same in which there are the input files, be careful if you want to exercute the program twice or more times at the same time, since it will take the output file also as input file.


\section{Modules and Packages}

\textit{Biopython} is the principal package used, as well as \textit{sys}, \textit{os}, \textit{re}, \textit{argparse} and the self-created modules \textit{homodimers} and \textit{heterodimers}. 

\subsection{Biopython}

Biopython is the main open-source collection of tools written in Python to work with biological data. From Biopython we take the following subpackages:

\begin{itemize}
 \item Bio.PDB, to work with PDB files
 \item Bio.pairwise2, to align protein sequences one by one
 \item Bio.SubsMat, to import Substitution Matrices to score the alignment
\end{itemize}


\subsection{sys}

Sys package is the System-specific parameters and functions. This package is used to read the arguments in the command line (sys.argv) and to have access to the three channels of communication with the computer: the \textit{standard in} (sys.stdin), the \textit{standard out} (sys.stdout) and the \textit{standard error} (sys.stderr).

\subsection{os}

Os package is the Miscellaneous operating system interfaces. With this package, the program can access synonimous commands of the shell, allowing the program to, for example, change working directory (\textit{cd} in shell, \textit{os.chdir} in python). 
This package is usefull to call command lines from the system but without consuming as much CPU.

\subsection{re}

Re package in the Regular expression operations package. It allows to work with regular expressions with python. In the program, it is used to find PDB and FASTA files, searching the extension of FASTA (.fa, .fasta) and PDB (.pdb) as regular expression at the end of the files.


\subsection{argparse}

Argparse package is the Parser for command-line options, arguments annd sub-commands. This packages allows to include the options for the user. The options included in the program are described in section \textit{Options and Arguments}.

\subsection{Homodimers and Heterodimers}

Homodimers and Heterodimers are self-created modules which analyze the structures depending on the input files. If the pair of chains in the interacting files are the same (A-A), then it is considered \texitit{homodimer}. If the pair of chains in the interacting files are different (A-B) but all the files contain this heterodimer (files are A-B, A-B...), then it is considered \textit{repeated heterodimer}. Finally, if the pair of chains in the interacting files are different and all interacting files are different (A-B, B-C...), then it is considered \textit{heterodimer}. For homodimers and repeated heterodimers, homodimers module is used. For heterodimers, heterodimers module is used.

\subsubsection{Homodimers module}

\subsubsection{Heterodimers module}

Heterodimers module first extract all the sequences of the chains and align them with pairwise alignment and stores the percentage of identity (\%id). Those chains with a \%id greater than 99\% are assumed to be the same and then, they are going to be superimposed. 
Once we know which chains must be superimposed, we do so taking one as reference chain and the other as moving chain. Then the coordinates are updated and the reference chain is deleted in order not to have repeated superimposed chains. 

\section{Workflow}

First of all, we take the input files provided by the user. We create an empty PDB object with Bio.PDB module and we fill this empty structure with the atoms that belong to protein chains and whose sequence length is greater than 20 aminoacids (in order to avoid other ligands). \\\\
After that, we analyze the sequence of the PDB files to decide if we treat the interaction files as homodimers (A-A interactions), repeated heterodimers (A-B, A-B...) or distinct heterodimers (A-B, B-C...).
If the interactions are homodimers (A-A) or repeated heterodimers (A-B, A-B...), then we use the standalone module \textit{homodimers.py}. If the interactions are distinct heterodimers (A-B, B-C...), then we use the module \textit{heterodimers.py}. You can see the workflow of each module in section \textit{Modules and Packages}, subsection \textit{Homodimers and Heterodimers}.\\\\
Finally, the program creates a PDB file with the final structure. If the option visualize (-vz, --visualize) is activated, it opens the PDB with chimera (see section \textit{Options and Arguments} for requirements of this option).

\section{Restrictions and Limitations of the Program}

\subsection{Homodimer module}

\subsection{Heterodimer module}

One of the main restrictions of this module is that all chains must be able to be followed. That is, we must be able to build a path joining all the subunits (A-B, B-C, C-D...). If it is not like that (A-B, C-D... for example), the program will crash. This is because one chain will be used as reference chain, and the other as moving chain. If it only appears once, then it is not possible to build the structure.


\chapter{How to use the program}

\section{Requirements}

The main program requires for \textit{Biopython} package and auxiliary modules \textit{homodimers.py} and \textit{heterodimers.py}.

\section{Options and Arguments}

In the program the following arguments are available:

\begin{itemize}
 \item -i, --input; it takes the directory as input. By default, it takes the current directory.
 \item -o, --ouput; it takes the directory and the filename of the PDB file. By default, it creates a file called \textit{output.pdb} in the current directory.
 \item -s, --sequence; it takes the directory and the filename of the FASTA file in which there is the sequence of the macro-complex. By default, it takes de value \textit{None} and runs the program with default parameters.
% \item -vz, --visualize; if this option is active, at the end of the script it opens the output file with Chimera. It will only work if Chimera is installed and callable by python interpreter.
 \item -v, --verbose; if this option is active, it prints the log to the standard error. 
\end{itemize}

All options are not forced to be, but it is highly recommended to use them, specially those related to the input/ouput files.

\section{Logging}

Activating the option for verbose (-v, --verbose), it prints to the standard error the log of the program. The point that are verbosed and the message if it is all right are the followings:

\begin{itemize}
 \item 
\end{itemize}

\section{Running the program}

To run the program in the terminal:\\ \texttt{\$ python3 main.py -i [input files] -o [output file]} \\\\ You can activate as many options as you would like:\\ \texttt{\$ python3 main.py -i [input files] -o [output file] -v }\\\\



\chapter{Analysis of examples}

\section{Tested examples}

We tested two %three
examples. One which is an heterotrimer (heterotrimeric G protein) and one repeated dimer (the example provided by Python teacher).

\subsection{Heterotrimer}

The example used is an heterotrimeric G protein, whose PDB id is 3AH8. We split it into two subunits (chain A - chain B and chain B - chain G).  

\section{Generalisation of the program}




\chapter{Discussion of the project}


\chapter{Conclusions}



%\chapter{Appendix}

%\section{Script}

%\section{README}





\end{document}
