\documentclass[a4paper,12pt]{report}
%\documentclass[a4paper,10pt]{scrartcl}

\usepackage[utf8]{inputenc}

\title{Reconstruction of a macro-complex using interacting subunits}
\author{Lydia Fortea \and Juan Luis Melero}
\date{}

\pdfinfo{%
  /Title    (Reconstruction of a macro-complex using interacting subunits)
  /Author   (Lydia Fortea \and Juan Luis Melero)
  /Creator  ()
  /Producer ()
  /Subject  (Structural Bioinformatics \and Introduction to Python)
  /Keywords (modelling, reconstruction, macro-complex, structural alignment, structural bioinformatics)
}

\begin{document}
\maketitle
\tableofcontents{}

\chapter{Background}


\section{Protein Structure}

Protein structure is the three-dimensional arrangement of atoms in a protein. 
This structure is made due to the possible rotation of the polypeptide chain and the nature of the residue of the protein.
In agreement with the Second Law of Termodynamics, the protein will get fold when the minimal energy is found, although they do not try all the possibilities since it is not possible in a short time (Levinthal's Paradox).
Therefore, the protein structure will be determined by the environment (usually water) and the nature of the residues (hydrophobic, hydrophilic, acidic, basic, aromatic, etc.).
//
//
Depending on the level of observation of the protein structure, there are four different classifications of the structure. These are primary, secondary, terciary and quaternary structure.
For some level there are also structural features such as supersecondary structures, motifs, domains... which they are not a level of structure but stable and conserved enough to be useful to classify.


\subsection{Primary Structure}

Protein primary structure is the linear sequence of aminoacids in a peptide or protein. This level does not refer to the three-dimensional but the chemical composition of the protein and despite of this,
the aminoacid sequence of the protein determines the fold. This rule is called the Anfinsen's dogma or the Thermodynamic hypothesis. By convention, the primary structure starts from the N-terminal end and finishes at C-terminal end, which is the order in which they are synthesized and it can be inferred from the DNA or RNA sequence.


\subsection{Secondary Structure}

\subsection{Terciary Structure and Domains}


\subsection{Quaternary Structure and Protein-Protein Interaction}







\section{Structural Bioinformatics}

\subsection{Structural Superimposition}

\subsection{Modelling}

\subsection{Control parameters of the model}




\chapter{Algorithm}

\section{Inputs and Outputs}

\section{Modules}

\section{Workflow}

\section{Restrictions and Limitations of the Algorithm}




\chapter{Analysis of examples}

\section{Proved examples}

\section{Generalisation of the program}




\chapter{Discussion of the project}


\chapter{Conclusions}



\chapter{Appendix}

\section{Script}

\section{README}





\end{document}
